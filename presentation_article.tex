% Options for packages loaded elsewhere
\PassOptionsToPackage{unicode}{hyperref}
\PassOptionsToPackage{hyphens}{url}
%
\documentclass[
  11pt,
  ignorenonframetext,
]{article}
\usepackage{pgfpages}
\setbeamertemplate{caption}[numbered]
\setbeamertemplate{caption label separator}{: }
\setbeamercolor{caption name}{fg=normal text.fg}
\beamertemplatenavigationsymbolsempty
% Prevent slide breaks in the middle of a paragraph
\widowpenalties 1 10000
\raggedbottom
\setbeamertemplate{part page}{
  \centering
  \begin{beamercolorbox}[sep=16pt,center]{part title}
    \usebeamerfont{part title}\insertpart\par
  \end{beamercolorbox}
}
\setbeamertemplate{section page}{
  \centering
  \begin{beamercolorbox}[sep=12pt,center]{part title}
    \usebeamerfont{section title}\insertsection\par
  \end{beamercolorbox}
}
\setbeamertemplate{subsection page}{
  \centering
  \begin{beamercolorbox}[sep=8pt,center]{part title}
    \usebeamerfont{subsection title}\insertsubsection\par
  \end{beamercolorbox}
}
\AtBeginPart{
  \frame{\partpage}
}
\AtBeginSection{
  \ifbibliography
  \else
    \frame{\sectionpage}
  \fi
}
\AtBeginSubsection{
  \frame{\subsectionpage}
}
\usepackage{amsmath,amssymb}
\usepackage{setspace}
\usepackage{iftex}
\ifPDFTeX
  \usepackage[T1]{fontenc}
  \usepackage[utf8]{inputenc}
  \usepackage{textcomp} % provide euro and other symbols
\else % if luatex or xetex
  \usepackage{unicode-math} % this also loads fontspec
  \defaultfontfeatures{Scale=MatchLowercase}
  \defaultfontfeatures[\rmfamily]{Ligatures=TeX,Scale=1}
\fi
\usepackage{lmodern}
\usefonttheme{serif} % use mainfont rather than sansfont for slide text

\ifPDFTeX\else
  % xetex/luatex font selection
    \setmainfont[]{STSong}
\fi
% Use upquote if available, for straight quotes in verbatim environments
\IfFileExists{upquote.sty}{\usepackage{upquote}}{}
\IfFileExists{microtype.sty}{% use microtype if available
  \usepackage[]{microtype}
  \UseMicrotypeSet[protrusion]{basicmath} % disable protrusion for tt fonts
}{}
\makeatletter
\@ifundefined{KOMAClassName}{% if non-KOMA class
  \IfFileExists{parskip.sty}{%
    \usepackage{parskip}
  }{% else
    \setlength{\parindent}{0pt}
    \setlength{\parskip}{6pt plus 2pt minus 1pt}}
}{% if KOMA class
  \KOMAoptions{parskip=half}}
\makeatother
\usepackage{xcolor}
\geometry{a4paper}
\newif\ifbibliography
\usepackage{color}
\usepackage{fancyvrb}
\newcommand{\VerbBar}{|}
\newcommand{\VERB}{\Verb[commandchars=\\\{\}]}
\DefineVerbatimEnvironment{Highlighting}{Verbatim}{commandchars=\\\{\}}
% Add ',fontsize=\small' for more characters per line
\newenvironment{Shaded}{}{}
\newcommand{\AlertTok}[1]{\textcolor[rgb]{1.00,0.00,0.00}{\textbf{#1}}}
\newcommand{\AnnotationTok}[1]{\textcolor[rgb]{0.38,0.63,0.69}{\textbf{\textit{#1}}}}
\newcommand{\AttributeTok}[1]{\textcolor[rgb]{0.49,0.56,0.16}{#1}}
\newcommand{\BaseNTok}[1]{\textcolor[rgb]{0.25,0.63,0.44}{#1}}
\newcommand{\BuiltInTok}[1]{\textcolor[rgb]{0.00,0.50,0.00}{#1}}
\newcommand{\CharTok}[1]{\textcolor[rgb]{0.25,0.44,0.63}{#1}}
\newcommand{\CommentTok}[1]{\textcolor[rgb]{0.38,0.63,0.69}{\textit{#1}}}
\newcommand{\CommentVarTok}[1]{\textcolor[rgb]{0.38,0.63,0.69}{\textbf{\textit{#1}}}}
\newcommand{\ConstantTok}[1]{\textcolor[rgb]{0.53,0.00,0.00}{#1}}
\newcommand{\ControlFlowTok}[1]{\textcolor[rgb]{0.00,0.44,0.13}{\textbf{#1}}}
\newcommand{\DataTypeTok}[1]{\textcolor[rgb]{0.56,0.13,0.00}{#1}}
\newcommand{\DecValTok}[1]{\textcolor[rgb]{0.25,0.63,0.44}{#1}}
\newcommand{\DocumentationTok}[1]{\textcolor[rgb]{0.73,0.13,0.13}{\textit{#1}}}
\newcommand{\ErrorTok}[1]{\textcolor[rgb]{1.00,0.00,0.00}{\textbf{#1}}}
\newcommand{\ExtensionTok}[1]{#1}
\newcommand{\FloatTok}[1]{\textcolor[rgb]{0.25,0.63,0.44}{#1}}
\newcommand{\FunctionTok}[1]{\textcolor[rgb]{0.02,0.16,0.49}{#1}}
\newcommand{\ImportTok}[1]{\textcolor[rgb]{0.00,0.50,0.00}{\textbf{#1}}}
\newcommand{\InformationTok}[1]{\textcolor[rgb]{0.38,0.63,0.69}{\textbf{\textit{#1}}}}
\newcommand{\KeywordTok}[1]{\textcolor[rgb]{0.00,0.44,0.13}{\textbf{#1}}}
\newcommand{\NormalTok}[1]{#1}
\newcommand{\OperatorTok}[1]{\textcolor[rgb]{0.40,0.40,0.40}{#1}}
\newcommand{\OtherTok}[1]{\textcolor[rgb]{0.00,0.44,0.13}{#1}}
\newcommand{\PreprocessorTok}[1]{\textcolor[rgb]{0.74,0.48,0.00}{#1}}
\newcommand{\RegionMarkerTok}[1]{#1}
\newcommand{\SpecialCharTok}[1]{\textcolor[rgb]{0.25,0.44,0.63}{#1}}
\newcommand{\SpecialStringTok}[1]{\textcolor[rgb]{0.73,0.40,0.53}{#1}}
\newcommand{\StringTok}[1]{\textcolor[rgb]{0.25,0.44,0.63}{#1}}
\newcommand{\VariableTok}[1]{\textcolor[rgb]{0.10,0.09,0.49}{#1}}
\newcommand{\VerbatimStringTok}[1]{\textcolor[rgb]{0.25,0.44,0.63}{#1}}
\newcommand{\WarningTok}[1]{\textcolor[rgb]{0.38,0.63,0.69}{\textbf{\textit{#1}}}}
\usepackage{longtable,booktabs,array}
\usepackage{calc} % for calculating minipage widths
\usepackage{caption}
% Make caption package work with longtable
\makeatletter
\def\fnum@table{\tablename~\thetable}
\makeatother
\usepackage{graphicx}
\makeatletter
\def\maxwidth{\ifdim\Gin@nat@width>\linewidth\linewidth\else\Gin@nat@width\fi}
\def\maxheight{\ifdim\Gin@nat@height>\textheight\textheight\else\Gin@nat@height\fi}
\makeatother
% Scale images if necessary, so that they will not overflow the page
% margins by default, and it is still possible to overwrite the defaults
% using explicit options in \includegraphics[width, height, ...]{}
\setkeys{Gin}{width=\maxwidth,height=\maxheight,keepaspectratio}
% Set default figure placement to htbp
\makeatletter
\def\fps@figure{htbp}
\makeatother
\ifLuaTeX
  \usepackage{luacolor}
  \usepackage[soul]{lua-ul}
\else
  \usepackage{soul}
  \makeatletter
  \let\HL\hl
  \renewcommand\hl{% fix for beamer highlighting
    \let\set@color\beamerorig@set@color
    \let\reset@color\beamerorig@reset@color
    \HL}
  \makeatother
\fi
\setlength{\emergencystretch}{3em} % prevent overfull lines
\providecommand{\tightlist}{%
  \setlength{\itemsep}{0pt}\setlength{\parskip}{0pt}}
\setcounter{secnumdepth}{-\maxdimen} % remove section numbering
\ifLuaTeX
  \usepackage{selnolig}  % disable illegal ligatures
\fi
\usepackage{bookmark}
\IfFileExists{xurl.sty}{\usepackage{xurl}}{} % add URL line breaks if available
\urlstyle{same}
\hypersetup{
  pdftitle={一次编写;多变输出},
  pdfauthor={CHEN, Xiaoqiang(陈孝强)},
  hidelinks,
  pdfcreator={LaTeX via pandoc}}

\title{一次编写;多变输出}
\author{CHEN, Xiaoqiang(陈孝强)}
\date{07 一月 2025}
\institute{粑粑柑革命共同体}

\begin{document}
\frame{\titlepage}

\setstretch{1.25}
\section{General information beamer}\label{general-information-beamer}

\subsection{Themes, fonts, etc. beamer}\label{themes-fonts-etc.-beamer}

\begin{itemize}
\tightlist
\item
  I use default \textbf{pandoc} themes.
\item
  This presentation is made with \textbf{Frankfurt} theme and
  \textbf{beaver} color theme.
\item
  I like \textbf{professionalfonts} font scheme.
\end{itemize}

\subsection{Links beamer}\label{links-beamer}

\begin{itemize}
\tightlist
\item
  Matrix of beamer themes:
  \url{https://hartwork.org/beamer-theme-matrix/}
\item
  Font themes:
  \href{http://www.deic.uab.es/~iblanes/beamer_gallery/index_by_font.html}{http://www.deic.uab.es/\textasciitilde iblanes/beamer\emph{gallery/index}by\_font.html}
\item
  Nerd Fonts: \url{https://nerdfonts.com}
\end{itemize}

\section{Formatting beamer}\label{formatting-beamer}

\subsection{Text formatting beamer}\label{text-formatting-beamer}

Normal text. \emph{Italic text} and \textbf{bold text}. \st{Strike out}
is supported.

\subsection{Notes beamer}\label{notes-beamer}

\begin{quote}
This is a note.

\begin{quote}
Nested notes are not supported. And it continues.
\end{quote}
\end{quote}

\subsection{Blocks}\label{blocks}

\subsubsection{This is a block A}\label{this-is-a-block-a}

\begin{itemize}
\tightlist
\item
  Line A
\item
  Line B
\end{itemize}

\subsubsection{}\label{section}

New block without header.

\subsubsection{This is a block B.
beamer}\label{this-is-a-block-b.-beamer}

\begin{itemize}
\tightlist
\item
  Line C
\item
  Line D
\end{itemize}

\subsection{Listings}\label{listings}

Listings out of the block.

\begin{Shaded}
\begin{Highlighting}[]
\CommentTok{\#!/bin/bash}
\BuiltInTok{echo} \StringTok{"Hello world!"}
\BuiltInTok{echo} \StringTok{"line"}
\end{Highlighting}
\end{Shaded}

\subsubsection{Listings in the block.
beamer}\label{listings-in-the-block.-beamer}

\begin{Shaded}
\begin{Highlighting}[]
\CommentTok{\#!/bin/bash}
\BuiltInTok{echo} \StringTok{"Hello world!"}
\BuiltInTok{echo} \StringTok{"line"}
\end{Highlighting}
\end{Shaded}

\subsection{Table beamer}\label{table-beamer}

\begin{longtable}[]{@{}lrc@{}}
\toprule\noalign{}
\textbf{Item} & \textbf{Description} & \textbf{Q-ty} \\
\midrule\noalign{}
\endhead
\bottomrule\noalign{}
\endlastfoot
Item A & Item A description & 2 \\
Item B & Item B description & 5 \\
Item C & N/A & 100 \\
\end{longtable}

\subsection{Single picture}\label{single-picture}

This is how we insert picture. Caption is produced automatically from
the alt text.

\begin{verbatim}
![Aleph 0](img/aleph0.png) 
\end{verbatim}

\begin{figure}
\centering
\includegraphics{img/aleph0.png}
\caption{Aleph 0}
\end{figure}

\subsection{Two or more pictures in a
raw}\label{two-or-more-pictures-in-a-raw}

Here are two pictures in the raw. We can also change two pictures size
(height or width).

\subsubsection{}\label{section-1}

\begin{verbatim}
![](img/aleph0.png){height=10%}\ ![](img/aleph0.png){height=30%} 
\end{verbatim}

\includegraphics[width=\textwidth,height=0.1\textheight]{img/aleph0.png}~\includegraphics[width=\textwidth,height=0.3\textheight]{img/aleph0.png}

\subsection{Lists}\label{lists}

\begin{enumerate}
\tightlist
\item
  Idea 1
\item
  Idea 2

  \begin{itemize}
  \tightlist
  \item
    genius idea A
  \item
    more genius 2
  \end{itemize}
\item
  Conclusion
\end{enumerate}

\subsection{Two columns of equal
width}\label{two-columns-of-equal-width}

Left column text.

Another text line.

\begin{itemize}
\tightlist
\item
  Item 1.
\item
  Item 2.
\item
  Item 3.
\end{itemize}

\subsection{Two columns of with 40:60
split}\label{two-columns-of-with-4060-split}

Left column text.

Another text line.

\begin{itemize}
\tightlist
\item
  Item 1.
\item
  Item 2.
\item
  Item 3.
\end{itemize}

\subsection{Three columns with equal
split}\label{three-columns-with-equal-split}

Left column text.

Another text line.

Middle column list:

\begin{enumerate}
\tightlist
\item
  Item 1.
\item
  Item 2.
\end{enumerate}

Right column list:

\begin{itemize}
\tightlist
\item
  Item 1.
\item
  Item 2.
\end{itemize}

\subsection{Three columns with 30:40:30
split}\label{three-columns-with-304030-split}

Left column text.

Another text line.

Middle column list:

\begin{enumerate}
\tightlist
\item
  Item 1.
\item
  Item 2.
\end{enumerate}

Right column list:

\begin{itemize}
\tightlist
\item
  Item 1.
\item
  Item 2.
\end{itemize}

\subsection{Two columns: image and
text}\label{two-columns-image-and-text}

\includegraphics[width=\textwidth,height=0.5\textheight]{img/aleph0.png}

Text in the right column.

List from the right column:

\begin{itemize}
\tightlist
\item
  Item 1.
\item
  Item 2.
\end{itemize}

\subsection{Two columns: image and
table}\label{two-columns-image-and-table}

\includegraphics[width=\textwidth,height=0.5\textheight]{img/aleph0.png}

\begin{longtable}[]{@{}lc@{}}
\toprule\noalign{}
\textbf{Item} & \textbf{Option} \\
\midrule\noalign{}
\endhead
\bottomrule\noalign{}
\endlastfoot
Item 1 & Option 1 \\
Item 2 & Option 2 \\
\end{longtable}

\subsection{Fancy layout beamer}\label{fancy-layout-beamer}

\subsubsection{Proposal}\label{proposal}

\begin{itemize}
\tightlist
\item
  Point A
\item
  Point B
\end{itemize}

\subsubsection{Pros}\label{pros}

\begin{itemize}
\tightlist
\item
  Good
\item
  Better
\item
  Best
\end{itemize}

\subsubsection{Cons beamer}\label{cons-beamer}

\begin{itemize}
\tightlist
\item
  Bad
\item
  Worse
\item
  Worst
\end{itemize}

\subsubsection{Conclusion beamer}\label{conclusion-beamer}

\begin{itemize}
\tightlist
\item
  Let's go for it!
\item
  No way we go for it!
\end{itemize}

\end{document}
