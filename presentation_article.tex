% Options for packages loaded elsewhere
\PassOptionsToPackage{unicode}{hyperref}
\PassOptionsToPackage{hyphens}{url}
\PassOptionsToPackage{dvipsnames,svgnames,x11names}{xcolor}
\documentclass[
  11pt,
]{article}
\usepackage{xcolor}
\usepackage[a4paper]{geometry}
\usepackage{amsmath,amssymb}
\setcounter{secnumdepth}{-\maxdimen} % remove section numbering
\usepackage{iftex}
\ifPDFTeX
  \usepackage[T1]{fontenc}
  \usepackage[utf8]{inputenc}
  \usepackage{textcomp} % provide euro and other symbols
\else % if luatex or xetex
  \usepackage{unicode-math} % this also loads fontspec
  \defaultfontfeatures{Scale=MatchLowercase}
  \defaultfontfeatures[\rmfamily]{Ligatures=TeX,Scale=1}
\fi
\usepackage{lmodern}
\ifPDFTeX\else
  % xetex/luatex font selection
    \setmainfont[]{Noto Serif CJK SC}
\fi
% Use upquote if available, for straight quotes in verbatim environments
\IfFileExists{upquote.sty}{\usepackage{upquote}}{}
\IfFileExists{microtype.sty}{% use microtype if available
  \usepackage[]{microtype}
  \UseMicrotypeSet[protrusion]{basicmath} % disable protrusion for tt fonts
}{}
\makeatletter
\@ifundefined{KOMAClassName}{% if non-KOMA class
  \IfFileExists{parskip.sty}{%
    \usepackage{parskip}
  }{% else
    \setlength{\parindent}{0pt}
    \setlength{\parskip}{6pt plus 2pt minus 1pt}}
}{% if KOMA class
  \KOMAoptions{parskip=half}}
\makeatother
\usepackage{listings}
\newcommand{\passthrough}[1]{#1}
\lstset{defaultdialect=[5.3]Lua}
\lstset{defaultdialect=[x86masm]Assembler}
\usepackage{longtable,booktabs,array}
\usepackage{calc} % for calculating minipage widths
\usepackage{caption}
% Make caption package work with longtable
\makeatletter
\def\fnum@table{\tablename~\thetable}
\makeatother
\usepackage{graphicx}
\makeatletter
\newsavebox\pandoc@box
\newcommand*\pandocbounded[1]{% scales image to fit in text height/width
  \sbox\pandoc@box{#1}%
  \Gscale@div\@tempa{\textheight}{\dimexpr\ht\pandoc@box+\dp\pandoc@box\relax}%
  \Gscale@div\@tempb{\linewidth}{\wd\pandoc@box}%
  \ifdim\@tempb\p@<\@tempa\p@\let\@tempa\@tempb\fi% select the smaller of both
  \ifdim\@tempa\p@<\p@\scalebox{\@tempa}{\usebox\pandoc@box}%
  \else\usebox{\pandoc@box}%
  \fi%
}
% Set default figure placement to htbp
\def\fps@figure{htbp}
\makeatother
\setlength{\emergencystretch}{3em} % prevent overfull lines
\providecommand{\tightlist}{%
  \setlength{\itemsep}{0pt}\setlength{\parskip}{0pt}}
\usepackage{bookmark}
\IfFileExists{xurl.sty}{\usepackage{xurl}}{} % add URL line breaks if available
\urlstyle{same}
\hypersetup{
  pdftitle={一次输入,多重输出},
  pdfauthor={CHEN,Xiaoqiang(陈孝强)},
  colorlinks=true,
  linkcolor={Maroon},
  filecolor={Maroon},
  citecolor={Blue},
  urlcolor={red},
  pdfcreator={LaTeX via pandoc}}

\title{一次输入,多重输出}
\author{CHEN,Xiaoqiang(陈孝强)}
\date{2024-01-01}

\begin{document}
\maketitle

{
\hypersetup{linkcolor=}
\setcounter{tocdepth}{3}
\tableofcontents
}
\section{介绍}\label{ux4ecbux7ecd}

\subsection{Themes, fonts, etc.}\label{themes-fonts-etc.}

\begin{itemize}
\tightlist
\item
  I use default \textbf{pandoc} themes.
\item
  This presentation is made with \textbf{Frankfurt} theme and
  \textbf{beaver} color theme.
\item
  I like \textbf{professionalfonts} font scheme.
\end{itemize}

\subsection{Links}\label{links}

\begin{itemize}
\tightlist
\item
  Matrix of beamer themes:
  \url{https://hartwork.org/beamer-theme-matrix/}
\item
  Font themes:
  \url{http://www.deic.uab.es/~iblanes/beamer_gallery/index_by_font.html}
\item
  Nerd Fonts: \url{https://nerdfonts.com}
\end{itemize}

\section{Formatting beamer}\label{formatting-beamer}

\subsection{Text formatting}\label{text-formatting}

Normal text. \emph{Italic text} and \textbf{bold text}.

\subsection{Notes}\label{notes}

\begin{quote}
This is a note.

\begin{quote}
Nested notes are not supported. And it continues.
\end{quote}
\end{quote}

\subsection{Blocks}\label{blocks}

\subsubsection{This is a block A}\label{this-is-a-block-a}

\begin{itemize}
\tightlist
\item
  Line A
\item
  Line B
\end{itemize}

New block without header.

\subsubsection{This is a block B.}\label{this-is-a-block-b.}

\begin{itemize}
\tightlist
\item
  Line C
\item
  Line D
\end{itemize}

\subsection{Listings}\label{listings}

Listings out of the block.

\begin{lstlisting}[language=sh]
#!/bin/bash
echo "Hello world!"
echo "line"
\end{lstlisting}

\subsubsection{Listings in the block.}\label{listings-in-the-block.}

\begin{lstlisting}[language=sh]
#!/bin/bash
echo "Hello world!"
echo "line"
\end{lstlisting}

\subsection{Table}\label{table}

\begin{longtable}[]{@{}lrc@{}}
\toprule\noalign{}
\textbf{Item} & \textbf{Description} & \textbf{Q-ty} \\
\midrule\noalign{}
\endhead
\bottomrule\noalign{}
\endlastfoot
Item A & Item A description & 2 \\
Item B & Item B description & 5 \\
Item C & N/A & 100 \\
\end{longtable}

\subsection{Single picture}\label{single-picture}

This is how we insert picture. Caption is produced automatically from
the alt text.

\begin{lstlisting}
![Aleph 0](img/aleph0.png) 
\end{lstlisting}

\begin{figure}
\centering
\pandocbounded{\includegraphics[keepaspectratio]{img/aleph0.png}}
\caption{Aleph 0}
\end{figure}

\subsection{Two or more pictures in a
raw}\label{two-or-more-pictures-in-a-raw}

Here are two pictures in the raw. We can also change two pictures size
(height or width).

\begin{lstlisting}
![](img/aleph0.png){height=10%}\ ![](img/aleph0.png){height=30%} 
\end{lstlisting}

\includegraphics[width=\linewidth,height=0.1\textheight,keepaspectratio]{img/aleph0.png}~\includegraphics[width=\linewidth,height=0.3\textheight,keepaspectratio]{img/aleph0.png}

\subsection{Lists}\label{lists}

\begin{enumerate}
\def\labelenumi{\arabic{enumi}.}
\tightlist
\item
  Idea 1
\item
  Idea 2

  \begin{itemize}
  \tightlist
  \item
    genius idea A
  \item
    more genius 2
  \end{itemize}
\item
  Conclusion
\end{enumerate}

\subsection{Two columns of equal
width}\label{two-columns-of-equal-width}

Left column text.

Another text line.

\begin{itemize}
\tightlist
\item
  Item 1.
\item
  Item 2.
\item
  Item 3.
\end{itemize}

\subsection{Two columns of with 40:60
split}\label{two-columns-of-with-4060-split}

Left column text.

Another text line.

\begin{itemize}
\tightlist
\item
  Item 1.
\item
  Item 2.
\item
  Item 3.
\end{itemize}

\subsection{Three columns with equal
split}\label{three-columns-with-equal-split}

Left column text.

Another text line.

Middle column list:

\begin{enumerate}
\def\labelenumi{\arabic{enumi}.}
\tightlist
\item
  Item 1.
\item
  Item 2.
\end{enumerate}

Right column list:

\begin{itemize}
\tightlist
\item
  Item 1.
\item
  Item 2.
\end{itemize}

\subsection{Three columns with 30:40:30
split}\label{three-columns-with-304030-split}

Left column text.

Another text line.

Middle column list:

\begin{enumerate}
\def\labelenumi{\arabic{enumi}.}
\tightlist
\item
  Item 1.
\item
  Item 2.
\end{enumerate}

Right column list:

\begin{itemize}
\tightlist
\item
  Item 1.
\item
  Item 2.
\end{itemize}

\subsection{Two columns: image and
text}\label{two-columns-image-and-text}

\includegraphics[width=\linewidth,height=0.5\textheight,keepaspectratio]{img/aleph0.png}

Text in the right column.

List from the right column:

\begin{itemize}
\tightlist
\item
  Item 1.
\item
  Item 2.
\end{itemize}

\subsection{Two columns: image and
table}\label{two-columns-image-and-table}

\includegraphics[width=\linewidth,height=0.5\textheight,keepaspectratio]{img/aleph0.png}

\begin{longtable}[]{@{}lc@{}}
\toprule\noalign{}
\textbf{Item} & \textbf{Option} \\
\midrule\noalign{}
\endhead
\bottomrule\noalign{}
\endlastfoot
Item 1 & Option 1 \\
Item 2 & Option 2 \\
\end{longtable}

\subsection{Fancy layout}\label{fancy-layout}

\subsubsection{Proposal}\label{proposal}

\begin{itemize}
\tightlist
\item
  Point A
\item
  Point B
\end{itemize}

\subsubsection{Pros}\label{pros}

\begin{itemize}
\tightlist
\item
  Good
\item
  Better
\item
  Best
\end{itemize}

\subsubsection{Cons}\label{cons}

\begin{itemize}
\tightlist
\item
  Bad
\item
  Worse
\item
  Worst
\end{itemize}

\subsubsection{Conclusion}\label{conclusion}

\begin{itemize}
\tightlist
\item
  Let's go for it!
\item
  No way we go for it!
\end{itemize}

\end{document}
